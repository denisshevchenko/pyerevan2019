\documentclass[aspectratio=169]{beamer}

\usepackage{fontspec}
\usepackage{listings}
\usepackage{color}
\usepackage{enumitem}
\usepackage{marvosym}
\usepackage{tikz}
\usepackage[skins]{tcolorbox}
\usepackage[normalem]{ulem}

\usetheme{parz} % Parz means "simple" in Armenian language.

\newcommand{\soutthick}[1]{%
    \renewcommand{\ULthickness}{2.4pt}%
       \sout{#1}%
    \renewcommand{\ULthickness}{.4pt}% Resetting to ulem default
}

\begin{document}

\begin{frame}
    \begin{columns}[c]
        \begin{column}[c]{15em}
            \begin{tikzpicture}
                \node[ circle
                     , draw
                     , inner sep=1.6cm
                     , fill overzoom image=img/me.jpg
                     ] {};
            \end{tikzpicture}

            \vspace{2.4em}

            \begin{minipage}{12.5em}
                \includegraphics[scale=0.30]{img/py_yerevan_left_aligned.png}
                \normalsize
                    August 2019
            \end{minipage}
        \end{column}

        \begin{column}[c]{15em}
            \Large
                Denis Shevchenko

            \vspace{1.4em}

            \large
                \begin{enumerate}[ leftmargin=0.78em
                                 , itemsep=0.3em
                                 , labelsep=0.1em
                                 , label=\color{gray}\Roman*
                                 ]
                    \item [\KeyWord{\Checkedbox}]\hspace{0.5em}Haskell Developer at IOHK
                    \item [\KeyWord{\Checkedbox}]\hspace{0.5em}Co-founder of ruHaskell
                    \item [\KeyWord{\Checkedbox}]\hspace{0.5em}Code since 2005
                \end{enumerate}

            \vspace{1.4em}

            \begin{minipage}{1.5em}
                \includegraphics[scale=0.02]{img/twitter.png}
            \end{minipage}
            \begin{minipage}{6em}
                \large
                    @dshevchenko\_biz
            \end{minipage}

            \vspace{5em}
        \end{column}
    \end{columns}
\end{frame}

\begin{frame}
    \HUGE
        A practical introduction\newline to \KeyWord{FP} in Python
\end{frame}

\begin{frame}
    \centering
        \HUGE
            FP?
\end{frame}

\begin{frame}
    \HUGE
        FP --- \KeyWord{rewrite} your code?
\end{frame}

\begin{frame}[fragile,t]
    \vspace{0.8em}
    \Huge{OOP's \KeyWord{for}}

    \begin{lstlisting}[language=Python]
names = ['David', 'Karen', 'Ruben']
secret_names = []
for i in range(len(names)):
    secret_names.append(hash(names[i]))
    \end{lstlisting}
\end{frame}

\begin{frame}[fragile,t]
    \vspace{0.8em}
    \Huge{FP's \KeyWord{map}}

    \begin{lstlisting}[language=Python]
names = ['David', 'Karen', 'Ruben']
secret_names = map(hash, names)
    \end{lstlisting}
\end{frame}

\begin{frame}
    \HUGE
        \hspace{2.35em}\KeyWord{rethink}\newline
        FP --- \soutthick{rewrite} your code!
\end{frame}

\begin{frame}[fragile,t]
    \begin{lstlisting}[language=Python]
a = 0

def inc():
    a += 1

def checker():
    if a > 12:
        print("Enough!")
    else:
        print("Too small yet")
    \end{lstlisting}
\end{frame}

\begin{frame}[fragile,t]
    \begin{lstlisting}[language=Python]
def inc(a):
    a += 1
    return a

def checker(a):
    if a > 12:
        print("Enough!")
    else:
        print("Too small yet")
    \end{lstlisting}
\end{frame}






\begin{frame}
    \centering
        \HUGE{What's \KeyWord{next}?}
\end{frame}

\begin{frame}
    \Huge{«A practical introduction}

    \Huge{\hspace{0.59em}to functional programming»}

    \vspace{1em}

    \Huge{bit.do/\KeyWord{fpPython}}
\end{frame}

\begin{frame}
    \centering
        \Huge{Thank you!}

        \HUGE{\KeyWord{Questions?}}
\end{frame}

\end{document}
