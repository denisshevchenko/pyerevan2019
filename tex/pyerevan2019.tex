\documentclass[aspectratio=169]{beamer}

\usepackage{fontspec}
\usepackage{listings}
\usepackage{color}
\usepackage{enumitem}
\usepackage{marvosym}
\usepackage{tikz}
\usepackage{tikz-cd}
\usepackage[skins]{tcolorbox}
\usepackage[normalem]{ulem}

\usetheme{parz} % Parz means "simple" in Armenian language.

\newcommand{\soutthick}[1]{%
    \renewcommand{\ULthickness}{5.4pt}%
       \sout{#1}%
    \renewcommand{\ULthickness}{.4pt}% Resetting to ulem default
}

\begin{document}

\begin{frame}
    \begin{columns}[c]
        \begin{column}[c]{15em}
            \begin{tikzpicture}
                \node[ circle
                     , draw
                     , inner sep=1.6cm
                     , fill overzoom image=img/me.jpg
                     ] {};
            \end{tikzpicture}

            \vspace{2.4em}

            \begin{minipage}{12.5em}
                \includegraphics[scale=0.30]{img/py_yerevan_left_aligned.png}
                \normalsize
                    August 2019
            \end{minipage}
        \end{column}

        \begin{column}[c]{15em}
            \Large
                Denis Shevchenko

            \vspace{1.4em}

            \large
                \begin{enumerate}[ leftmargin=0.78em
                                 , itemsep=0.3em
                                 , labelsep=0.1em
                                 , label=\color{gray}\Roman*
                                 ]
                    \item [\KeyWord{\Checkedbox}]\hspace{0.5em}Haskell Developer at IOHK
                    \item [\KeyWord{\Checkedbox}]\hspace{0.5em}Co-founder of ruHaskell
                    \item [\KeyWord{\Checkedbox}]\hspace{0.5em}Code since 2005
                \end{enumerate}

            \vspace{1.4em}

            \begin{minipage}{1.5em}
                \includegraphics[scale=0.02]{img/twitter.png}
            \end{minipage}
            \begin{minipage}{6em}
                \large
                    @dshevchenko\_biz
            \end{minipage}

            \vspace{5em}
        \end{column}
    \end{columns}
\end{frame}

\begin{frame}
    \HUGE
        A practical introduction\newline to \KeyWord{FP} in Python
\end{frame}

\begin{frame}
    \centering
        \HUGE
            FP?
\end{frame}

\begin{frame}
    \HUGE
        FP --- \KeyWord{rewrite} your code!
\end{frame}

\begin{frame}[fragile,t]
    \vspace{0.8em}
    \Huge{OOP's \KeyWord{for}}

    \begin{lstlisting}[language=Python]
names = ['David', 'Karen', 'Ruben']
secrets = []
for i in range(len(names)):
    secrets.append(hash(names[i]))
    \end{lstlisting}
\end{frame}

\begin{frame}[fragile,t]
    \vspace{0.8em}
    \Huge{FP's \KeyWord{map}}

    \begin{lstlisting}[language=Python]
names = ['David', 'Karen', 'Ruben']
secrets = map(hash, names)
    \end{lstlisting}
\end{frame}

\begin{frame}
    \HUGE
        \hspace{2.32em}\KeyWord{rethink}\newline
        FP --- \soutthick{rewrite} your code!
\end{frame}

\begin{frame}[fragile,t]
    \begin{lstlisting}[language=Python]
def inc():
    global a
    a += 1

def check():
    if a > 12:
        print("Enough!")
    \end{lstlisting}
\end{frame}

\begin{frame}[fragile]
    \begin{lstlisting}[language=Python]
a = 3

inc()
check()
    \end{lstlisting}
\end{frame}

\begin{frame}
    \centering
        \Huge{OOP $\to$ FP}

        \vspace{1em}

        \HUGE{\KeyWord{Reduce} mutable state!}
\end{frame}

\begin{frame}[fragile,t]
    \begin{lstlisting}[language=Python]
def inc(a):
    return(a + 1)

def check(a):
    if a > 12:
        return "Enough!"
    \end{lstlisting}
\end{frame}

\begin{frame}[fragile]
    \begin{lstlisting}[language=Python]
print(check(inc(3)))
    \end{lstlisting}
\end{frame}

\begin{frame}
    \centering
        \Huge{\KeyWord{Compose} your functions}
\end{frame}

\begin{frame}[fragile]
    \begin{figure}[!h]
        \centering
            \begin{tikzcd}[transform canvas={scale=2.4}]
                A \arrow[thick, r, "f"]
                  \arrow[thick, rr, out=-35, in=215, swap, awesome, "g\ \ \circ\ \ f"]
                &[2.3em] B \arrow[thick, r, "g"]
                &[2.3em] C
            \end{tikzcd}
    \end{figure}
\end{frame}

\begin{frame}[fragile]
    \begin{figure}[!h]
        \centering
            \begin{tikzcd}[transform canvas={scale=2.4}]
                A \arrow[thick, r, "inc"]
                  \arrow[thick, rr, out=-35, in=215, swap, awesome, "check\ \ \circ\ \ inc"]
                &[2.3em] B \arrow[thick, r, "check"]
                &[2.3em] C
            \end{tikzcd}
    \end{figure}
\end{frame}

\begin{frame}[fragile,t]
    \begin{lstlisting}[language=Python]
from infix import shift_infix as infix

@infix
def then(f, g):
    return lambda x: g(f(x))
    \end{lstlisting}
\end{frame}

\begin{frame}[fragile,t]
    \begin{lstlisting}[language=Python]
do = inc <<then>> check
    \end{lstlisting}
\end{frame}

\begin{frame}[fragile,t]
    \begin{lstlisting}[language=Python]
do = inc <<then>> check

do_all = do <<then>> print
    \end{lstlisting}
\end{frame}

\begin{frame}[fragile,t]
    \begin{lstlisting}[language=Python]
do = inc <<then>> check

do_all = do <<then>> print

...

do_all(3)
    \end{lstlisting}
\end{frame}

\begin{frame}
    \centering
        \Huge{Use \KeyWord{fair} constants}
\end{frame}

\begin{frame}[fragile,t]
    \begin{lstlisting}[language=Python]
PI = 3.14159265

...

def calcCircLength(r: float) -> float:
    return 2 * PI * r
    \end{lstlisting}
\end{frame}

\begin{frame}[fragile,t]
    \begin{lstlisting}[language=Python]
import cmath

...

def circle_length(r: float) -> float:
    return 2 * cmath.pi * r
    \end{lstlisting}
\end{frame}

\begin{frame}[fragile,t]
    \begin{lstlisting}[language=Python]
import cmath

cmath.pi = 3.129 # Bad joke!

def circle_length(r: float) -> float:
    return 2 * cmath.pi * r
    \end{lstlisting}
\end{frame}

\begin{frame}[fragile,t]
    \begin{lstlisting}[language=Python]
def pi() -> float:
    return 3.14159265

...

def circle_length(r: float) -> float:
    return 2 * pi() * r
    \end{lstlisting}
\end{frame}

\begin{frame}
    \centering
        \HUGE{Math?!}
\end{frame}

\begin{frame}
    \centering
        \Huge
            $M = (S, \bullet, e)$
\end{frame}

\begin{frame}
    \Huge
        \begin{enumerate}[ leftmargin=1.5em
                         , itemsep=1em
                         , labelsep=0.5em
                         , label=\color{darkelectricblue}\Roman*
                         ]
            \item $S = \{...\}$
            \item $\bullet : S \times S \to S$
            \item $\exists e \in S \mid e \bullet a = a \bullet e = a$
        \end{enumerate}
\end{frame}

\begin{frame}
    \centering
        \begin{figure}[b]
            \includegraphics[scale=0.51]{img/what.jpg}
        \end{figure}
\end{frame}

\begin{frame}[fragile,t]
    \begin{lstlisting}[language=Python]
https_prefix = "https://"
base_url     = "istc.am"
secure_url   = https_prefix + base_url
    \end{lstlisting}
\end{frame}

\begin{frame}[fragile,t]
    \begin{lstlisting}[language=Python]
https_prefix = "https://"
base_url     = "istc.am"
secure_url   = https_prefix + base_url
    \end{lstlisting}

    \vspace{2em}

    \centering
        \Huge
            Monoid --- \KeyWord{idea}!
\end{frame}

\begin{frame}[fragile,t]
    \begin{lstlisting}[language=Python]
https_prefix = "https://"
base_url     = "istc.am"
secure_url   = https_prefix + base_url
    \end{lstlisting}

    \vspace{2em}

    \centering
        \Huge
            Take \KeyWord{existent}, create \KeyWord{new}.
\end{frame}

\begin{frame}
    \centering
        \HUGE{So...}
\end{frame}

\begin{frame}
    \centering
        \HUGE{\KeyWord{Reduce} mutable state!}
\end{frame}

\begin{frame}
    \Huge
        \begin{enumerate}[ leftmargin=1.5em
                         , itemsep=1em
                         , labelsep=0.5em
                         , label=\color{darkelectricblue}\Roman*
                         ]
            \item \KeyWord{Compose} your functions
            \item Use \KeyWord{fair} constants
            \item Don't be \KeyWord{afraid} of math
        \end{enumerate}
\end{frame}

\begin{frame}
    \centering
        \HUGE{What's \KeyWord{next}?}
\end{frame}

\begin{frame}
    \Huge{«A practical introduction}

    \Huge{\hspace{0.59em}to functional programming»}

    \vspace{1em}

    \Huge{bit.do/\KeyWord{fpPython}}
\end{frame}

\begin{frame}
    \centering
        \Huge{Thank you!}

        \HUGE{\KeyWord{Questions?}}
\end{frame}

\end{document}
